%%%%%%%%%%%%%%%%%%%%%%%%%%%%%%%%%%%%%%%
% Wenneker Resume/CV
% LaTeX Template
% Version 1.1 (19/6/2016)
%
% This template has been downloaded from:
% http://www.LaTeXTemplates.com
%
% Original author:
% Frits Wenneker (http://www.howtotex.com) with extensive modifications by 
% Vel (vel@LaTeXTemplates.com)
%
% License:
% CC BY-NC-SA 3.0 (http://creativecommons.org/licenses/by-nc-sa/3.0/
%
%%%%%%%%%%%%%%%%%%%%%%%%%%%%%%%%%%%%%%

%----------------------------------------------------------------------------------------
%	PACKAGES AND OTHER DOCUMENT CONFIGURATIONS
%----------------------------------------------------------------------------------------

\documentclass[a4paper,12pt]{memoir} % Font and paper size

%%%%%%%%%%%%%%%%%%%%%%%%%%%%%%%%%%%%%%%%%
% Wenneker Resume/CV
% Structure Specification File
% Version 1.1 (19/6/2016)
%
% This file has been downloaded from:
% http://www.LaTeXTemplates.com
%
% Original author:
% Frits Wenneker (http://www.howtotex.com) with extensive modifications by 
% Vel (vel@latextemplates.com)
%
% License:
% CC BY-NC-SA 3.0 (http://creativecommons.org/licenses/by-nc-sa/3.0/)
%
%%%%%%%%%%%%%%%%%%%%%%%%%%%%%%%%%%%%%%%%%

%----------------------------------------------------------------------------------------
%	PACKAGES AND OTHER DOCUMENT CONFIGURATIONS
%----------------------------------------------------------------------------------------

\usepackage{XCharter} % Use the Bitstream Charter font
\usepackage[utf8]{inputenc} % Required for inputting international characters
\usepackage[T1]{fontenc} % Output font encoding for international characters

\usepackage[top=1cm,left=1cm,right=1cm,bottom=1cm]{geometry} % Modify margins

\usepackage{graphicx} % Required for figures

\usepackage{flowfram} % Required for the multi-column layout

\usepackage{url} % URLs

\usepackage[usenames,dvipsnames]{xcolor} % Required for custom colours

\usepackage{tikz} % Required for the horizontal rule

\usepackage{enumitem} % Required for modifying lists
\setlist{noitemsep,nolistsep} % Remove spacing within and around lists

\setlength{\columnsep}{\baselineskip} % Set the spacing between columns

% Define the left frame (sidebar)
\newflowframe{0.23\textwidth}{\textheight}{0pt}{0pt}[left]
\newlength{\LeftMainSep}
\setlength{\LeftMainSep}{0.235\textwidth}
\addtolength{\LeftMainSep}{1\columnsep}
 
% Small static frame for the vertical line
\newstaticframe{1.5pt}{\textheight}{\LeftMainSep}{0pt}
 
% Content of the static frame with the vertical line
\begin{staticcontents}{1}
\hfill
\tikz{\draw[loosely dotted,color=RoyalBlue,line width=1pt,yshift=0](0,0) -- (0,\textheight);}
\hfill\mbox{}
\end{staticcontents}
 
% Define the right frame (main body)
\addtolength{\LeftMainSep}{1.pt}
\addtolength{\LeftMainSep}{1\columnsep}
\newflowframe{0.7\textwidth}{\textheight}{\LeftMainSep}{0pt}[main01]

\pagestyle{empty} % Disable all page numbering

\setlength{\parindent}{0pt} % Stop paragraph indentation

%----------------------------------------------------------------------------------------
%	NEW COMMANDS
%----------------------------------------------------------------------------------------

\newcommand{\userinformation}[1]{\renewcommand{\userinformation}{#1}} % Define a new command for the CV user's information that goes into the left column

\newcommand{\cvheading}[1]{{\Large\bfseries\color{RoyalBlue} #1} \par\vspace{.6\baselineskip}} % New command for the CV heading
\newcommand{\cvsubheading}[1]{{\large\bfseries #1} \bigbreak} % New command for the CV subheading

\newcommand{\Sep}{\vspace{1em}} % New command for the spacing between headings
\newcommand{\SmallSep}{\vspace{0.5em}} % New command for the spacing within headings

\newcommand{\aboutme}[2]{ % New command for the about me section
\textbf{\color{RoyalBlue} #1}~~\small#2\par\Sep
}
	
\newcommand{\CVSection}[1]{ % New command for the headings within sections
{\large\textbf{#1}}\par
\SmallSep % Used for spacing
}

\newcommand{\CVItem}[2]{ % New command for the item descriptions
\textbf{\color{RoyalBlue} #1}\par\small{
#2}
\SmallSep % Used for spacing
}

\newcommand{\bluebullet}{\textcolor{RoyalBlue}{$\circ$}~~} % New command for the blue bullets
\newcommand{\smallBlueBullet}{\footnotesize\textcolor{RoyalBlue}{$\circ$}~~} % New command for the blue bullets
 % Include the file specifying document layout and packages

\userinformation{
\begin{flushright}
\footnotesize % Smaller font size
\url{me@juanvaldivia.com} \\ % Your email address
\footnotesize (+351) 968 640 681 \\ % Your phone number
\Sep
\end{flushright}

\aboutme{\footnotesize About Me}
{\footnotesize
I am a quality driven software engineer 
that enjoys crafting efficient and \\
user-friendly 
solutions specially as an enabler to other 
colleagues.
 I am currently set in Lisbon, Portugal.
 }
\Sep


\CVSection{Skills}
%------------------------------------------------
\CVItem{Programming languages and others}
{\begin{tabular}{p{0.2\textwidth}}
\smallBlueBullet Java 
\smallBlueBullet Go \\
\smallBlueBullet Typescript \\
\smallBlueBullet Agile development \\
\smallBlueBullet Microservices \\
\smallBlueBullet Serverless \\
\smallBlueBullet Github Actions \\
\smallBlueBullet AWS
\smallBlueBullet Azure \\

\end{tabular}}
%------------------------------------------------

\CVItem{Spoken languages}{
\begin{tabular}{p{0.2\textwidth}}
    \smallBlueBullet \textit{Portuguese} Native\\ 
    \smallBlueBullet \textit{Spanish} Native\\
    \smallBlueBullet \textit{English} (C2 level)\\ 
    \smallBlueBullet \textit{French} (C1 level)\\ 
\end{tabular}}

\Sep

\CVSection{Interests}

%------------------------------------------------

\CVItem{Professional}{
I'm interested on topics like software quality, software design and functional languages idioms. 
I keep up to date by attending software 
meetups. I usually \\ pickup technical reads from books and internet articles.
}

%------------------------------------------------

\CVItem{Personal}{ 
 I like hiking and doing some photography at it. I also like to collect coins from different times and places.
}


\vfill

}

%----------------------------------------------------------------------------------------

\begin{document}

\userinformation % Print your information in the left column

\framebreak % End of the first column

\cvheading{Juan Valdivia} 
\cvsubheading{Software Engineer}

\CVSection{Education}
%------------------------------------------------
\CVItem{2011 -2015, ISCTE-University Intitute of Lisbon}{BS in Computer Science and some MS in Computer Science classes}

\Sep

\CVSection{Experience}
%------------------------------------------------
\CVItem{Jan 2019 - present, \textit{Software Engineer}, Carlsberg Group IIT}{
\vspace{1pt}
\textbf{Built and launched Cadi (Carlsberg Group's sales team application)}
\begin{itemize}
	\item Made available to 7 European markets with a small multi functional team
	\item Helped to lay down and document ground rules for development practices
        \item Solution design and development for several micro-services and integration deployed in cloud environment in Europe and Asia
\end{itemize}

\textbf{In 2022, worked in the Platform Team as one of the lead developers.}
\begin{itemize}
	\item Building common services with API first practices
	\item Restructuring CI/CD pipelines to trunk based branching
	\item Refactoring older services to serve higher request demand and allow scalability.
	\item Evangelizing \textit{innersource} practices throughout the teams
\end{itemize}

\textbf{As of 2023, I movde to the Carlshop team, Carlsberg Group's flagship e-commerce platform.}
\begin{itemize}
	\item Security patching to aging platform (SAP Hybris)
	\item Assisting migration of project to SAP Commerce Cloud (SAP Saas)
	\item Refactoring base functionalities for better maintainability
\end{itemize}
}

%------------------------------------------------

\CVItem{Oct 2017 - Dec 2018, \textit{Software Engineer}, STEF-IT}{
\textbf{Worked building and maintaining customer management tool for French company specialized in cold chain transportation and logistic}
\begin{itemize}
	\item Major language migration implying great rework on UI (GWT)
	\item Corrective maintenance on search engine and SOA platform.
	\item Taking under my responsibility new younger developers
\end{itemize}
}
%------------------------------------------------

\CVItem{Aug 2016 - Aug 2017, \textit{Software Engineer}, TIMWE Group}{
\textbf{Worked on middle-ware to enhance mobile carriers and content partners capabilities for mobile subscriptions and promotion services.}
\begin{itemize}
	\item Built back-office for mobile VAS\footnote{Value Added Services} runtime and did custom integrations to different AAA\footnote{Authorization,Authentication,Accounting} services.
	\item Successfully migrated over 750K users from 6 different countries to work in the new platform
	\item Highly involved with clients to gather and discuss business and technical requirements

\end{itemize}
}

%------------------------------------------------

\Sep % Extra whitespace after the end of a major section

\CVSection{Side Projects}

%------------------------------------------------

\CVItem{2019, \textit{Carlsberg Group}, 12h Hackathon}{Top 3. Business analysis and showcase of a solution to handle an upcoming real life business threat. Built a POC in Javascript}

\CVItem{2018, \textit{STEF-IT}, 24h Hackathon}{Awarded first place. Demo application in Javascript that tackled workplace safety education practices around STEF-IT trucks, lifts and warehouses }

\CVItem{2018, \textit{Novabase GTE (now Axians)}, DevOps Learning}{DevOps culture investigation. Interviewed success cases. Collected articles and books to form internal community. }

\end{document}
